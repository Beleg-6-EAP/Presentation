\section{Klassische Enterprise-Architektur}

%%%%%%%%%%%%%%%%%%%%%%%%%% Service-oriented Architecture %%%%%%%%%%%%%%%%%%%%%%%%%%
\begin{frame}{Service-oriented Architecture}
\begin{itemize}
  \item Service Provider: Bietet einen spezifischen Dienst an
  \item Service Bus: Dienst, das die Orchestierung der Komunikation zwischen Service Consumer und Service Provider gewährleistet.
  \item Service Consumer: Nutzt einen bereitgestelten Dienst, dieses kann einen End-User oder ein anderer Dienst sein.
  \item Service Registry: Dient als zentrales Repository, die Informationen über die verfügbare Services speichert (Interface-Vertag + Endpoint).
\end{itemize}
\begin{frame}{Service-oriented Architecture: Struktur}

    \begin{figure}[!h]
        \centering
        \includegraphics[scale=0.55]{imglib/soa/soa.pdf}
        \caption{Aufbau der Service-oriented Architecture}
        \label{fig:soa}
    \end{figure}
\end{frame}

\begin{frame}{Service-oriented Architecture: Beispiel E-Commerce I}
    \begin{itemize}
        \item \texttt{OrderService}: Ein Dienst, der die Erstellung von Bestellungen übernimmt und dabei den Versand- sowie den Bezahlvorgang automatisch einleitet.in Dienst, das die Erstellung von Bestellung übernimmt. Versandvorgang und Bezahlvorgang werden initiert.
        \item \texttt{PaymentService}: Dienst verantwortlich für die Abwicklung und Überprüfung von Zahlungen
        \item \texttt{ShipmentService}: Dienst zuständig für den Versandprozesses
    \end{itemize}
\end{frame}

\begin{frame}{Service-oriented Architecture: Beispiel E-Commerce II}
    \begin{figure}[!h]
        \centering
        \includegraphics[scale=0.5]{imglib/soa/soa-example.pdf}
        \caption{E-Commerce-Beispiel mit Service-oriented Architecture}
        \label{fig:soaecommerce}
    \end{figure}
\end{frame}

\begin{frame}{Service-oriented Architecture: Agilität}
    \begin{itemize}
         \end{itemize}
\end{frame}


