\section{Klassische Enterprise-Architektur}

%%%%%%%%%%%%%%%%%%%%%%%%%% Service-oriented Architecture %%%%%%%%%%%%%%%%%%%%%%%%%%

\begin{frame}{Service-oriented Architecture}
  \begin{itemize}
    \item Service: Dienst, das eine Funktionalität kapselt.
    Dieser kann aus mehreren Diensten bestehen
    \item Service Provider: Bietet einen spezifischen Dienst an
    \item Service Bus: Dienst, das die Kommunikation und Integration zwischen Service Consumer und Service Provider gewährleistet
    \item Service Consumer: Nutzt einen bereitgestellten Dienst, dieses kann einen End-User oder ein anderer Dienst sein
    \item Service Registry: Dient als zentrales Repository, die Informationen über die verfügbaren Services speichert (Interface-Vertrag und Endpoint)
  \end{itemize}
\end{frame}
\begin{frame}{Service-oriented Architecture: Struktur}
  \begin{figure}[!h]
    \centering
    \includegraphics[scale=0.55]{imglib/soa/soa}
    \caption{Aufbau der Service-oriented Architecture}
    \label{fig:soa}
  \end{figure}
\end{frame}

\begin{frame}{Service-oriented Architecture: Beispiel E-Commerce I}
  \begin{itemize}
    \item \texttt{OrderService}: Ein Dienst, der die Erstellung von Bestellungen übernimmt und dabei den Versand- sowie den Bezahlvorgang automatisch initiiert.
    \item \texttt{PaymentService}: Dienst verantwortlich für die Abwicklung von Zahlungen
    \item \texttt{ShipmentService}: Dienst zuständig für den Versandprozess
  \end{itemize}
\end{frame}

\begin{frame}{Service-oriented Architecture: Beispiel E-Commerce II}
  \begin{figure}[!h]
    \centering
    \includegraphics[scale=0.5]{imglib/soa/soa-example}
    \caption{E-Commerce-Beispiel mit Service-oriented Architecture}
    \label{fig:soaecommerce}
  \end{figure}
\end{frame}

\begin{frame}{Service-oriented Architecture: Agilität}
  \begin{itemize}
    \item Services können unabhängig voneinander parallel entwickelt werden
    \item Interfaces ermöglichen gute Kollaboration im Team
    \item Kundenanforderungen können kurzfristig umgesetzt werden
    \item Zeit- und Kosteneinsparungen durch die Wiederverwendung von Services
    \item Nachteil: Langfristig könnten sich Abhängigkeiten zwischen den Services entwickeln
    \end{itemize}
\end{frame}
