\section{Übungsaufgaben}

\begin{frame}{Übungsaufgabe I}
    Das E-Commerce-Beispiel aus der Einleitung soll um Nutzer-Authentifizierung erweitert werden.
    Sie haben sich zuvor für eine Microservice-Architektur entschieden und die in der Einleitung genannten Anforderungen implementiert.
    Die Authentifizierung wird in verschiedenen Komponenten benötigt.

    Erläutern Sie, wie Sie die Authentifizierung in die Architektur integrieren.
\end{frame}

\begin{frame}{Übungsaufgabe II}
    Das E-Commerce-Beispiel aus der Einleitung soll um Logging erweitert werden.
    Sie haben sich zuvor für eine cloud-native Event-Driven-Architektur entschieden und die in der Einleitung genannten Anforderungen implementiert.
    Das Logging wird in verschiedenen Komponenten benötigt.

    Erläutern Sie, wie Sie das Logging in die Architektur integrieren.

    Bedenken Sie dabei, dass Logs aus verschiedenen Komponenten möglicherweise zur Auswertung zusammengeführt werden müssen und somit die Reihenfolge von Logs relevant ist.
\end{frame}

\begin{frame}{Übungsaufgabe III}
    Das E-Commerce-Beispiel aus der Einleitung soll um ein E-Mail-Notifikationssystem erweitert werden.
    Dieses soll Nutzern E-Mails bei jeder Statusänderung einer ihrer Bestellungen zustellen.

    Untersuchen Sie für alle in diesem Papier betrachteten Architekturen, wie das E-Mail-Notifikationssystem in die bestehende Architektur integriert werden kann
    und welche Vor- und Nachteile diese Integration in die jeweilige Architektur mit sich bringt.

    Für welche der Architekturen ist die Integration des E-Mail-Notifikationssystems am einfachsten?
\end{frame}